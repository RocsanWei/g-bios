\section{Qemu虚拟环境}
本例采用 beagle 作为虚拟开发板,其他板子方法类似。
\begin{enumerate}
	\item 编译 g-bios: \\
	g-bios 中进入 th load 引导菜单可以通过长按键盘空格键进入,这在硬件平台上是完全可以做到的,但是由于虚拟机的启动比较快,很难做到(多次试验失败)。\\
	可以通过修改 g-bios th load 部分代码来达到目的。\\
	具体修改文件为:g-bios/th/base/main.c
	\lstset{language=[ANSI]C}
	\begin{lstlisting}[numbers=none]
		...
	37	// for(i = 0; i < 0x100; i++)
	38	while (1)
		...
	\end{lstlisting}
	修改OK后,按如下步骤编译:
	\begin{lstlisting}[language=sh,numbers=none]
	$ make beagle_defconfig
	$ make
	$ make install
	\end{lstlisting}
	若无错误,最后会出现如下提示:
	\begin{lstlisting}[language=sh,numbers=none]
	$ `bh/g-bios-bh.bin' -> `/maxwit/image/boot/g-bios-bh.bin'
	$ `th/g-bios-th.bin' -> `/maxwit/image/boot/g-bios-th.bin'
	\end{lstlisting}

	\item 制作 image:
	\begin{lstlisting}[language=csh,numbers=none]
	$ cp build/simulators/omap3530/create_nandflash.py /maxwit/image/boot
	$ ./create_nandflash.py g-bios-th.bin flash.img
	\end{lstlisting}
	在此 image 中只烧录 g-bios 的 th 部分。

	\item 运行:
	\begin{lstlisting}[language=csh,numbers=none]
	$ qemu-system-arm -M beagle -mtdblock flash.img -serial pty
	\end{lstlisting}
	得到如下信息:
	\begin{lstlisting}[numbers=none]
	0+1 records out
	1004 bytes (1.0 kB) copied, 4.2184e-05 s, 23.8 MB/s
	put bl1 into flash image done!
	char device redirected to /dev/pts/5
	\end{lstlisting}
	根据上面的启动信息,得到qemu用的通信串口为``/dev/pts/5'',
	敲入如下命令
	\begin{lstlisting}[language=csh,numbers=none]
	$ minicom -D /dev/pts/5
	\end{lstlisting}
	待 minicom 启动后敲任意键,即可得到 th 启动界面。
	接下来进入熟悉的 th 菜单界面,th 部分的 load 方式可以很容易的进行尝试。不清楚之处可以参考第二章烧录部分相关内容。

\end{enumerate}

\section{Others}
